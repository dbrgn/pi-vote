\documentclass{beamer}
\usetheme{malmoe}

\usepackage[latin1]{inputenc}
\usepackage{tikz}
\usepackage[underline=true,rounded corners=false]{pgf-umlsd}

\title{E-Voting done right, is it reliable?}
\author{Thomas Bruderer, Stefan Th�ni}
\date{\today}

\begin{document}

\frame{\titlepage}

\section{Outline}{}

\frame{\tableofcontents}

\section{Secret Ballot}

\frame
{
	\frametitle{Goal}
	
	\begin{itemize} 
		\item No individual vote must ever be revealed.
		\item The sum of votes must be published in a verifiable manner.
	\end{itemize}
	
	A contradiction?
}

\frame
{
	\frametitle{ElGamal public-key cryptosystem}

	System parameters
	\begin{itemize} 
		\item Safe prime $p= 2q + 1$ where $q$ prime
		\item Generator $g$ order $q$ element of $\mathbb{Z}_p^*$
	\end{itemize}
	
	\vspace{0.4cm}
	
	Everything that follows is in modular arithmetic!
	
	\vspace{0.4cm}
	
	Authority selects
	\begin{itemize} 
		\item Private key $x \in \mathbb{N}$
		\item Public key $h = g^x$
	\end{itemize}

}
\frame
{
	\frametitle{ElGamal public-key cryptosystem (cont'd.)}
	
	Voter selects
	\begin{itemize} 
		\item Vote $v \in \{0,1\}$
		\item Nonce $r \in \mathbb{N}$
	\end{itemize}

	\vspace{0.4cm}

	Encryption of $v$
	\begin{itemize}
		\item $(g^r, h^r f^v)$)
	\end{itemize}
	
	\vspace{0.4cm}

	Decryption
	\begin{itemize} 
		\item $(g^r)^x = (g^x)^r = h^r$
		\item $\frac{h^r f^v}{h^r} = f^v$
	\end{itemize}
}

\frame
{
	\frametitle{Homomorphic property}
	
	The sum of two encrypted votes $v_0$ and $v_1$ \\
	\vspace{0.6cm}
	$(g^{r_0}$, $h^{r_0} f^{v_0}) + (g^{r_1}$, $h^{r_1} f^{v_1}) =$ \\
	\vspace{0.6cm}
	$(g^{r_0} g^{r_1}$, $h^{r_0} f^{v_0} h^{r_1} f^{v_0}) =$ \\
	\vspace{0.6cm}
	$(g^{r_0 + r_1}, h^{r_0 + r_1} f^{v_0 + v_1})$ \\
	\vspace{0.6cm}
	yields the encrypted sum of $v_0$ and $v_1$!
}

\frame
{
	\frametitle{ElGamal homomorphic encryption - Results}
	
	\begin{itemize} 
		\item Can sum up encrypted votes
		\item Secure given the computational Diffie-Hellman assumption is true
		\item Therefore we require that the discrete logarithms are hard to compute
	\end{itemize}
}

\section{Division of Power}

\frame
{
	\frametitle{We need to trust. But in whom?}
	
	\begin{itemize}
		\item Before we saw that we need the key pair of an authority. But whom do we trust with this?
		\item No one individually. Rather we distribute the key among server authorities.
		\item Therefore we need a secret sharing scheme were $t$ of $n$ authorities can decrypt the result.
	\end{itemize}
}

\frame
{
	\frametitle{Shamir's Secret Sharing}
	
	Each authority $i$
	\begin{itemize}
		\item Creates random polynomial $P_i(x) \in \mathbb{Z}_q[x]$ with coefficients $c_{i,l}$ for $l \in 0...t-1$
		\item Evaluates for each authority $j$ as share $s_{i,j} = P_i(j)$
		\item Sends $s_{i,j}$ securely to $j$
		\item Calculates and publishes values $G_{i,l} = g^{c_{i,l}}$
	\end{itemize}
}

\frame
{
	\frametitle{Shamir's Secret Sharing (cont'd)}
	
	Each authority $j$
	\begin{itemize}
		\item Verify for each $i$ that $g^{s_{i,j}} = \prod\limits_{l=0}^{t-1}{(G_{i,l})^{i^k}}$
		\item Computes its secret key part as $x_i = \sum\limits_{i=0}^{n-1}{s_{i,j}}$
	\end{itemize}
	\vspace{0.8cm}
	Each voter
	\begin{itemize}
		\item Computes public key $h = \prod\limits_{i=0}^{n-1}{G_{i,0}}$
	\end{itemize}
}

\frame
{
	\frametitle{Shamir's Secret Sharing (cont'd)}
	
	Each authority $i$
	\begin{itemize}
		\item Sums up all encrypted votes $\sum{(g^{r_j}, h^{r_j} f^{v_j})} = (g^r, h^r f^v)$
		\item Computes and publishes partial deciphers $p_{i,k} = (g^r)^{x_i l_{i,k}}$ where $l_{i,k}$ are the Lagrange coefficients.
	\end{itemize}

	\vspace{0.3cm}

	Everyone
	\begin{itemize}
		\item Sums up all encrypted votes $\sum{(g^{r_j}, h^{r_j} f^{v_j})} = (g^r, h^r f^v)$
		\item Decrypt the sum of votes $g^r \prod{p_{i,k}} = (g^r)^x = h^r$ and $\frac{h^r f^v}{h^r} = f^v$ where the $v$ is the plain sum of votes.
	\end{itemize}
}

\frame
{
	\frametitle{Shamir's Secret Sharing - Results}
	
	\begin{itemize}
		\item Only $t$ out of $n$ authorities together can decrypt a vote
		\item Information theoretic security
	\end{itemize}
}

\section{Valid Ballot}

\frame
{
	\frametitle{Valid Ballot}
	
	What make a valid ballot?
	\begin{itemize}
		\item Each vote is $v \in \{0,1\}$ 
		\item The sum of all votes in the ballot is $\sum\limits_{o \in Options}{v_o} = m$ where $m$ is the allowed number of votes on this issue (typically $m = 1$)
	\end{itemize}
}

\frame
{
	\frametitle{Zero knowledge proof}
	
	\begin{itemize}
		\item A proof of knowledge: Peggy proves to Vic that she nows an $x$ such that $x$ has some property $P$
		\item In zero knowledge: Peggy proves this such that Vic does not learn anything about $x$ besides the fact that it satisfies $P$
	\end{itemize}
}

\frame
{
	\frametitle{Zero knowledge proof (cont'd)}
	
	Which means for Pi-Vote
	\begin{itemize}
		\item Remember encryption: $(g^r ,h^r f^v)$
		\item Proof to know an $r$ such that $(g^r, h^r f^v)$ where $v$ is the sum of votes in a ballot.
		\item Transformed to know $r = log_h f^v$
		\item Fits perfectly to Schnorr's protocol
	\end{itemize}
}

\frame
{
	\frametitle{Schnorr's Protocol}
	
	\begin{sequencediagram}
		\newinst{p}{Peggy}
		\newinst[4]{v}{Vic}

		\messself{p}{$u = \in_R \mathbb{Z}_n$}
		\messself{p}{$a = h^u$}
		\mess{p}{a}{v}
		\messself{v}{$c = \in_R \{0,1\}$}
		\mess{v}{c}{p}
		\messself{p}{$s = u + x c$}
		\mess{p}{r}{v}
		\messself{v}{$g^s = a {f^v}^c$?}
	\end{sequencediagram}
}

\section{Right to Vote}

\frame
{
	\frametitle{Who has the right to vote?}
	
	\begin{itemize}
		\item Certificates for voters and authorities
		\item Using RSA and SHA2 exclusively, plus AES for encryption
		\item Our own, independent public key infrastructure
		\item Strict rules: One root only, must have a valid revocation list at all times
	\end{itemize}
}

\end{document}